%%%%%%%%%%%%%%%%%%%%%%%%%%%%%%%%%%%%%%%%%%
% 第一行通常设置文档类型,有的模板会采用自定义类型
%%%%%%%%%%%%%%%%%%%%%%%%%%%%%%%%%%%%%%%%%%
\documentclass[twocolumn, 10pt, a4paper]{article}

%%%%%%%%%%%%%%%%%%%%%%%%%%%%%%%%%%%%%%%%%%
% 一般在开头导入内置/第三方包
%%%%%%%%%%%%%%%%%%%%%%%%%%%%%%%%%%%%%%%%%%

%-----------------------------------------
% 解决字体渲染问题
%-----------------------------------------
\usepackage[utf8]{inputenc} % allow utf-8 input
% for typesetting underscore as a character
\usepackage[T1]{fontenc}
\usepackage{amssymb}        % for using \mathbb


%-----------------------------------------
% 支持基本数学公式环境
%-----------------------------------------
\usepackage{amsmath}        % required by \begin{align}...\end{align}


%-----------------------------------------
% 渲染图表
%-----------------------------------------
\usepackage{booktabs}              % for using toprule, midrule, bottomrule
\usepackage{multirow}              % for merging multiple rows into one cell
\usepackage{graphicx}              % for using \resizebox
\usepackage[table]{xcolor}         % for using \rowcolor, \cellcolor

\usepackage{tablefootnote}         % for putting footnote below tables
\usepackage[labelfont=bf]{caption} % for making captions bold in figures and tables
\captionsetup[table]{skip=1pt}     % for controlling caption distance
\captionsetup[figure]{skip=1pt}    % for controlling caption distance


%-----------------------------------------
% 参考文献的引用
%-----------------------------------------
\usepackage{cite}           % change the citation format from [1,2,3] to [1-3]


%-----------------------------------------
% 其他
%-----------------------------------------
\usepackage{setspace}     % for using \setstretch (must be loaded before hyperref)


%-----------------------------------------
% 添加跳转超链接,高亮不同类型的超链接
%-----------------------------------------
\usepackage{hyperref}
\hypersetup{
    colorlinks=true,
    linkcolor=blue,
    filecolor=magenta,      
    urlcolor=black,
    citecolor=magenta,
    pdfpagemode=FullScreen,
}

%\usepackage{footnotebackref}  % 添加脚注的反向跳转


%-----------------------------------------
% 添加自定义命令(如 \var),便于反复使用
%-----------------------------------------

%-------------------
% (1)快速隐藏表格的某一列
%-------------------
\newcolumntype{H}{>{\setbox0=\hbox\bgroup}c<{\egroup}@{}}

%-------------------
% (2)专有名词/变量快速跳转到定义处
%-------------------
\usepackage{etoolbox} % for using \ifcsdef

% Shift-up the jump location of \hypertarget by one line
\makeatletter
  \newcommand{\linkdest}[1]{\Hy@raisedlink{\hypertarget{#1}{}}}
\makeatother
% Define a command to typeset a symbol and create a hyperlink (without color) to its first appearance
\newcounter{varsymbolcount}
\newcommand{\var}[1]{%
  \ifcsdef{symbol:#1}{%
    % Symbol has been defined before, create a hyperlink to its first appearance
    {\hypersetup{hidelinks}\hyperlink{symbol:#1}{#1}}%
  }{%
    % First appearance of the symbol, create a hyperlink target
    \stepcounter{varsymbolcount}%
    \linkdest{symbol:#1}{}#1%
    \global\expandafter\def\csname symbol:#1\endcsname{}% Mark symbol as defined
  }%
}


%-----------------------------------------
% 节省空间的技巧
%-----------------------------------------

%-------------------
% (1)减少公式、图片、表格的上下间距
%-------------------
% reduce the distance between floats on the top or the bottom and the text
 \setlength{\textfloatsep}{5pt plus 0.0pt minus 2.0pt}
 % reduce the distance between two floats
 \setlength{\floatsep}{5pt plus 0.0pt minus 2.0pt}
 % reduce the distance between two floats [h]
 \setlength{\intextsep}{5pt plus 0.0pt minus 2.0pt}

%-------------------
% (2)减少参考文献之间的间隔
%-------------------
\let\oldthebibliography\thebibliography
\let\endoldthebibliography\endthebibliography
\renewenvironment{thebibliography}[1]{
  \begin{oldthebibliography}{#1}
  \setlength{\itemsep}{-0.1em}
  \setlength{\parskip}{0.em}
}
{
  \end{oldthebibliography}
}

%-------------------
% (3)拉伸表格行/列间距
%-------------------

% 行间距
%\begin{table}
%   \setstretch{0.92}
%   ...
%\end{table}

% 列间距
%\begin{table}
%    \setlength{\tabcolsep}{3pt}
%\end{table}




%-----------------------------------------
% 定义 BSTcontrol,用于控制参考文献的作者列表长度
%-----------------------------------------
\makeatletter
\def\bstctlcite{\@ifnextchar[{\@bstctlcite}{\@bstctlcite[@auxout]}}
\def\@bstctlcite[#1]#2{\@bsphack
  \@for\@citeb:=#2\do{%
    \edef\@citeb{\expandafter\@firstofone\@citeb}%
    \if@filesw\immediate\write\csname #1\endcsname{\string\citation{\@citeb}}\fi}%
  \@esphack}
\makeatother



%%%%%%%%%%%%%%%%%%%%%%%%%%%%%%%%%%%%%%%%%%
% 填写标题、作者、单位等基本信息
%%%%%%%%%%%%%%%%%%%%%%%%%%%%%%%%%%%%%%%%%%


\title{A Good Title is Worth a Thousand Words}

\author{Three Zhang$^{1,2}$ \and Four Li$^1$ \and Five Wang$^2$}
\date{%
    $^1$Organization 1\\%
    $^2$Organization 2 has such a looooooooooooong name that it\\
    cannot properly fit in a single line
}

%%%%%%%%%%%%%%%%%%%%%%%%%%%%%%%%%%%%%%%%%%
% 正文从这里开始
%%%%%%%%%%%%%%%%%%%%%%%%%%%%%%%%%%%%%%%%%%


\begin{document}
%-----------------------------------------    
\bstctlcite{IEEEexample:BSTcontrol} % place this line right below \begin{document}
%-----------------------------------------


% 加了下面这一行才会显示标题、作者等信息
\maketitle

%-----------------------------------------
% 以下开始填写摘要
%-----------------------------------------

\begin{abstract}
    Don't cite references in abstract.
    Usually we don't write multiple paragraphs in abstract.
    Mind the word count limit in the paper template (can differ across conferences/journals)!
    \textbf{Basic structure:} (1) 1 sentence describing the background; (2) 1 ``however'' sentence stating current problems; (3) 2--3 sentences introducing your goal and contributions; (4) 1--2 sentences highlighting the experimental results or key findings.
\end{abstract}

%-----------------------------------------
% 以下开始填写正文实际内容(如不同章节、参考文献等)
%-----------------------------------------

\section{Introduction}
\label{sec:intro}

Define the \textbf{full names} of abbreviations (initials) when they appear for the first time!
Even if they have already been defined in \emph{Abstract}, they should be defined again in the main body (starting from \emph{Introduction}).
For example, convolutional neural networks (CNNs) and Transformer~\cite{Attention-Vaswani2017} can be combined to form new architectures.


\section{Related Works}
\label{sec:related}

In \texttt{AISHELL-3}~\cite{AISHELL3-Shi2021}, it is reported that ...


\section{Proposed Method}
\label{sec:proposed}

We propose a new model called \textsc{HahahahaNet}.
It takes as input encoded features extracted from the DNSMOS P.835~\cite{DNSMOS-Reddy2022} model\footnote{\url{https://github.com/microsoft/DNS-Challenge/blob/master/DNSMOS/DNSMOS/sig_bak_ovr.onnx}}.

\subsection{Problem formulation}
\label{ssec:problem}

Let $\mathbf{A} \in \mathbb{C}A^{F \times F}$ be the xxx matrix, where $F$ is the frequency dimension. We have ...
{\allowdisplaybreaks % This allows for breaking multiple equations into different columns
\begin{align}
    \mathbf{X} &= \mathbf{\Phi}^{\textsf{H}}\mathbf{U}^{\textsf{T}}\mathbf{A}^{-1} \quad\in\mathbb{C}^{T \times F} \label{eq:example2} \,, \\
    M_f &= \dfrac{1}{T} \left(\sum_{t=0}^{T-1} w_t |X|^2_{t,f}\right)^{0.5} \label{eq:example3} \,,
\end{align}
}% <------ include this % to avoid space indent in the sentence below
where \var{$M_f$} represents the estimated mask of the $f$-th frequency bin.

Jepsen \textit{et al.}~\cite{Study-Jepsen2025} found that training speech enhancement models with noisy signal labels based on scale-invariant signal-to-noise ratio (\var{SI-SNR})~\cite{SISNR-LeRoux2019} can lead to inferior performance.

\subsection{First method}
\label{ssec:method1}

CHiME-6~\cite{CHiME6-Watanabe2020}


\begin{figure*}
  \centering
  \includegraphics[width=0.88\textwidth]{assets/images/logo_audiocc.png}
  \captionsetup{labelfont=bf}
  \caption[overview]{Overview of the proposed model.}
  \label{fig:overview}
  \vspace{-0.5em}
\end{figure*}

\subsection{Second method}
\label{ssec:method2}

We use the mask \var{$M_f$} to refine the final prediction.


\section{Experiments}
\label{sec:exp}

We evaluate the model performance using the \var{SI-SNR} metric.


\subsection{Experimental setup}
\label{ssec:exp_setup}


\subsection{Results and analysis}
\label{ssec:exp_results}

{
% reduce the distance between columns in a table
%\setstretch{0.92}
\setlength{\tabcolsep}{3pt}
\begin{table}[t]
    \caption{Tag occurrences in the dataset.}
    \label{tab:data_tags}
    \centering
    \resizebox{1.0\columnwidth}{!}{% <---- don't forget this %
        \begin{tabular}{lc|l|cc} 
        \toprule
        \multicolumn{2}{c|}{\textbf{Tag}} & \textbf{Note} & \textbf{Fullset} & \textbf{Subset} \\
        \midrule
        \multirow{3}{*}{\shortstack[l]{Data\\Domain}} & real\_recording & recorded directly &  & 0 \\
        & enhanced & (suspiciously)\,processed by SE systems & 0 &  \\
        & synthetic & (suspiciously)\,synthetic speech & 0 &  \\
        \hline
        \rowcolor[HTML]{EEEEEE} & non\_speech & w/o any intelligible speech\tablefootnote{This means ...} &  &  \\
        \rowcolor[HTML]{EEEEEE} \multirow{-2}{*}{\shortstack[l]{Speech Type}} & read\_speech &  &  &  \\
       \bottomrule
       \end{tabular}% <---- don't forget this %
    }
\end{table}
}


\section{Conclusion}
\label{sec:conclusion}


\section*{Acknowledgment}

This work was supported by xxx.


%-----------------------------------------
% 一般在文末添加参考文献
%-----------------------------------------

\bibliographystyle{IEEEtran}
\bibliography{mybib}



\end{document}